\documentclass[acmcompsurv,acmnow]{acmtrans2m}

\newtheorem{theorem}{Theorem}[section]
\newtheorem{conjecture}[theorem]{Conjecture}
\newtheorem{corollary}[theorem]{Corollary}
\newtheorem{proposition}[theorem]{Proposition}
\newtheorem{lemma}[theorem]{Lemma}
\newdef{definition}[theorem]{Definition}
\newdef{remark}[theorem]{Remark}

\markboth{M. E. Locasto}{A Brief History of Language-Theoretic Security's Roots}

\title{A Brief History of LangSec's Roots}
        
\author{
MICHAEL E. LOCASTO\\
University of Calgary\\
and PML \\ 
The Internet
}

\begin{abstract}
An abstract.
\end{abstract}
            
\category{D.2.0}{General}{Protection Mechanisms}
\category{A.1}{Introductory and Survey}{Security}
\category{K.6.5}{Security and Protection}{}

\terms{Security}

\keywords{langsec, safety, weird machines}

\begin{document}
            
\begin{bottomstuff} 
Authors' addresses: 
M. Locasto. Department of Computer Science, University of Calgary, 2500 University Drive NW., Calgary, AB, T2N 1N4. Canada
\newline
\end{bottomstuff}
            
\maketitle

\section{Introduction}

This paper is an introduction and overview of my perspective on LangSec
which has received increasing attention over the past four years.

This section includes the main message: ``research work in this area
is defining the boundary between theoretically impossible problems and
the merely hard when trying to get algorithms (as expressed in real,
complex software systems) to deal with unanticipated and malformed
input and poorly specified input languages.''

List major crisis and motivation of the work in this area.

Outline the structure of the paper; how it is organized (not just an
enumeration, but rather why it is organized that way: what are the
logical or temporal relationships between the major work). Forecast the
timeline section.

\subsection{Timeline}

Contains a timeline of the work

history; overview; phase 1, 2, 3

\section{Themes}

What are the major themes in play here?

\begin{enumerate}
  \item relations to other areas
  \item systems work
  \item theory work
  \item tbd
\end{enumerate}

\section{Approach/Domain 1}
papers in this domain do X.
\section{Approach/Domain 2}

\section{Summary}

Major challenges.

Where are we headed?

FIN.

\bibliographystyle{acmtrans}
\bibliography{langsec}
\begin{received}
Received MM YYYY;
Revised MM YYYYY;
Accepted MM YYYYYY;
\end{received}

\end{document}


