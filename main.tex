\documentclass[acmcompsurv,acmnow]{acmtrans2m}

\newtheorem{theorem}{Theorem}[section]
\newtheorem{conjecture}[theorem]{Conjecture}
\newtheorem{corollary}[theorem]{Corollary}
\newtheorem{proposition}[theorem]{Proposition}
\newtheorem{lemma}[theorem]{Lemma}
\newdef{definition}[theorem]{Definition}
\newdef{remark}[theorem]{Remark}

\markboth{M. E. Locasto}{A Brief Rumination on Language-Theoretic
  Security's Roots}

\title{A Brief Rumination on LangSec's Roots}
        
\author{
MICHAEL E. LOCASTO\\
University of Calgary\\
and PML \\ 
The Internet
}

\begin{abstract}
An annotated bibliography of LangSec elements appearing in academic
information security papers.  This paper is an attempt at documenting
my encounters with ideas similar to LangSec before its specific
articulation circa 2011-2012 by Sassaman, Patterson, and Bratus.
\end{abstract}
            
\category{D.2.0}{General}{Protection Mechanisms}
\category{A.1}{Introductory and Survey}{Security}
\category{K.6.5}{Security and Protection}{}

\terms{Security}

\keywords{langsec, safety, weird machines}

\begin{document}
            
\begin{bottomstuff} 
Authors' addresses: 
M. Locasto
\newline
\end{bottomstuff}
            
\maketitle

\section{Introduction}

In my opinion, LangSec is one of the most interesting and compelling
ideas to emerge in computer and network systems security over the past
few years.  In this paper, I review a number of papers that have
LangSec elements or roots.  In many cases, a paper I highlight
identifies quite explicitly some LangSec or weird machine ideas, but
(as the terms had not yet been widely used or publicized) does not yet
explicitely identify those ideas as such.  As is the case with most
ideas whose time has come, there is a body of work and a movement of
intellectual themes from others in the field that complement and give
context to the later idea.

In this brief review, I attempt to list some of the ideas and papers I
personally encountered that relate to LangSec and in some cases
predate its articulation.  The papers I discuss below have a certain
bias toward intrusion detection and intrusion response (vulnerability
identification, exploit signature generation, etc.).

Research work in the area of Langec is defining the boundary between
theoretically impossible problems and the merely hard when trying to
get algorithms (as expressed in real, complex software systems) to
deal with unanticipated and malformed input and poorly specified input
languages.

%{\bf Outline} Outline the structure of the paper; how it is organized
%(not just an enumeration, but rather why it is organized that way:
%what are the logical or temporal relationships between the major
%work). Forecast the timeline section.

\section{Papers}

This section is a necessarily incomplete list of academic research
papers that touch on ideas, concepts, and themes related to
LangSec. There is no particular order to the listing of these papers;
it is essentially stream-of-recall. I may re-order them later.

It is worth reading the ``official'' LangSec and Weird Machine papers
from the LangSec website\footnote{\url{http://www.langsec.org/}}:

\begin{enumerate}
 \item ``The Halting Problems of Network Stack Insecurity'', Len Sassaman, Meredith L. Patterson, Sergey Bratus, Anna Shubina
 \item ``Exploit Programming: from Buffer Overflows to Weird Machines and Theory of Computation'', Sergey Bratus, Michael E. Locasto, Meredith L. Patterson, Len Sassaman, Anna Shubina
 \item ``A Patch for Postel's Robustness Principle'', Len Sassaman, Meredith L. Patterson, Sergey Bratus
 \item ``Beyond Planted Bugs in ``Trusting Trust'': The Input-Processing Frontier'' by Sergey Bratus, Trey Darley, Michael Locasto, Meredith L. Patterson, Rebecca ``.bx'' Shapiro, Anna Shubina
 \item ``Security Applications of Formal Language Theory'' by Len Sassaman, Meredith L. Patterson, Sergey Bratus, Michael E. Locasto, Anna Shubina [Dartmouth Computer Science Technical Report TR2011-709], published in IEEE Systems Journal, Volume 7, Issue 3, Sept. 2013
\end{enumerate}

{\bf The UCDavis ``Deriving'' paper}\cite{Crandall:2005:DUV:1102120.1102152}

Crandall et al.'s paper ``On Deriving Unknown Vulnerabilities from
Zero-day Polymorphic and Metamorphic Worm Exploits'' was one of the
first places I saw an attempt to meaningfully formalize the operation
of what Bratus later came to call ``weird machines''~\cite{XXX}.
Although this CCS 2005 paper is where I first saw the
``Alpha-Gamma-Upsilon'' model proposed by this UCDavis team, a closer
reading shows that the model itself was proposed in an earlier paper
from the same team \cite{XXX}.  This paper is also notable because it
fits squarely into current work circa 2005 dealing with automated
response to exploits by deriving a signature of the software {\em
  weakness} rather than the particular exploit itself.

{\bf ``Fractures'' paper}\cite{Crandall:2012:HVS:2413296.2413309}

Crandall and Oliveira collaborated to write an interesting NSPW paper
that attempts to explore the nature of vulnerabilities through the
analogy of holographs.  The key idea here is the interefering role
that abstraction plays across system boundaries, and how this designed
and manifest lack of knowledge on different sides of system layers
contributes to computationally unequal processing of information.

{\bf ``SQLPrevent by San-Tsai Sun (UBC)''}

One of the earliest instances of a language-based approach to
protecting against SQL injection attacks I encountered (circa
2007--2008) was by San-Tsai Sun at UBC's LERSSE lab.  Up until this
point in time, most SQL injection defenses I'd seen in practive during
my time as a web developer were based on character escaping (in the
Web language or library itself, and thus likely a flawed
implementation) or the use of stored procedures (underutilized,
particularly in casual and basic web programming).  In a series of
technical reports and a journal paper from 2010\cite{XXX}, San-Tsai
Sun describes SQLPrevent, a system .

In the context of discussing SQLPrevent's language-based filtering
approach to preventing SQLi, It is noteworthy that Patterson's
Dejector\footnote{\url{http://sourceforge.net/projects/libdejector/}}
library\footnote{\url{http://www.thesmartpolitenerd.com/code/dejector.html}}
(2005) formed part of initial inspiration for Sassaman and Patterson's
later exposition and naming of LangSec. More recently, Dan Kaminsky's
Interpolique from
2010\footnote{\url{http://dankaminsky.com/interpolique/}} seeks to
counter the ill effects of using string composition as a method of
building SQL queries (i.e., SQL queries are not simply concatenated
strings, but rather structured artifacts that should be built up in a
disciplined fashion, with operators that respect the types being
manipulated).

\subsection{Recognizing Network Traffic (IDS)}

The area of network traffic recognition, filtering, and analysis is a
rich area for witnessing the problems that LangSec so amply
identifies; in many instances, the myriad and seemingly ad hoc
problems that IDS, firewalls, Deep Packet Inspection, protocol
identification, and worm signature generation all encounter are given
a unifying explanation by LangSec: solving the Halting Problem at line
speed is a daydream.

Two classic papers discussing the difficulties of ``recognizing''
network traffic are ``IDS Evasion Attacks''~\cite{XXX} by Ptacek and
Newsham and ``Traffic Normalization'' (by Handley et al.)~\cite{XXX}.


{\bf ``Remotely Exploiting the PHY Layer''}

Goodspeed et al.~\cite{}'s WOOT 2011 paper about exploiting noise in
frame delimiters is a more recent example of the problems due to
recognition of network frames and packets.  This work demonstrates how
PHY-layer (layer 1) frames can be injected in traffic without local
physical access.  The technique works by embedding a valid layer 1
preamble into a packet and either relying on normal noise or possibly
induced noise (e.g., a high packet rate) to scramble the ``real''
frame synchronization at the receiver.

\begin{enumerate}
  \item  \url{http://travisgoodspeed.blogspot.ca/2011/09/remotely-exploiting-phy-layer.html}
  \item  \url{http://www.usenix.org/events/woot11/tech/final_files/Goodspeed.pdf}
  \item  \url{https://www.usenix.org/conference/woot11/packets-packets-orson-welles-band-signaling-attacks-modern-radios}
  \item \url{http://www.phrack.org/issues.html?issue=68&id=4&mode=txt} 
    (see 0x06, ``How I misunderstood digital radio; or, ``Weird machines'' are in radio, too!'' by M.Laphroaig pastor@phrack )
\end{enumerate}

This work was followed up by a presentation at WOOT 2014 exploring the
n-bit symbols that actually make up each ``character'' of a
synchronization preamble.

\subsection{Other Work}

Vulnerable Compliance (Geer)

G. Wurster, P. C. van Oorschot. The Developer is the Enemy. In
Proc. 2008 Workshop on New Security Paradigms, September 2008. pp
89-97.

{\bf ``Beyond Heuristics: Learning to Classify Vulnerabilities and Predict Exploits''}


\subsection{Selected Papers from IEEE LangSec 2015 Workshop}

Unparsers


\subsection{On the use of ``Context'' in Systems Security}

Papers related to intrusion detection are particularly adept at making
use of unstructured context in improving their detection rates or
lowering their reported false positive rates.

``Modeling System Calls for Intrusion Detection with Dynamic Window Sizes''

``Gray-Box Extraction of Execution Graphs for Anomay Detection''

``Anomaly Detection Using Call Stack Information''

``Environment-Sensitive Intrusion Detection''

``Exploiting Execution Context for the Detection of Anomalous System Calls''



\section{Links}


    http://www.darkreading.com/vulnerability/taming-bad-inputs-means-taking-aim-at-we/240152171
    http://programmingisterrible.com/post/42215715657/postels-principle-is-a-bad-idea
    Programming with Nothing: http://experthuman.com/programming-with-nothing
    Learning to classify vulns: http://dl.acm.org/citation.cfm?doid=1835804.1835821
    PHY layer hacking: http://2012.hackitoergosum.org/blog/schedule/talks#Strangeand
    Catastrophic backtracking in regular expressions http://t.co/KWVDhLyI
    From Buffer Overflows to Weird Machines
    Cyberpatterns
    The Halting Problems of Network Stack Insecurity
    Security Applications of Formal Language Theory
    Packets in Packets (Goodspeed)
    Vulnerable Compliance (Geer)
    IDS Evasion Attacks (Ptacek and Newsham)
    Traffic Normalization (Handley)
    Crandall CCS 2005
    http://www.isg.rhul.ac.uk/tls/
    travis goodspeed: ``Remotely Exploiting the PHY Layer''
        http://travisgoodspeed.blogspot.ca/2011/09/remotely-exploiting-phy-layer.html
        WOOT 2011 paper: http://www.usenix.org/events/woot11/tech/final_files/Goodspeed.pdf
    https://www.usenix.org/conference/woot11/packets-packets-orson-welles-band-signaling-attacks-modern-radios
        http://www.phrack.org/issues.html?issue=68&id=4&mode=txt (see 0x06, ``How I misunderstood digital radio; or, ``Weird machines'' are in radio, too!'' by M.Laphroaig pastor@phrack )

Misc:

    http://www.microsoft.com/typography/otspec/featuretags.htm
    evading AV: http://blog.endpoint.com/2013/01/evading-anti-virus-metasploit.html
    http://programmingisterrible.com/post/42432568185/how-to-parse-ruby
    packet of death: http://appliance.cloudshark.org/news/cloudshark-in-the-wild/intel-packet-of-death-capture/
    blocking content based on executable env: http://arstechnica.com/security/2013/01/firefox-to-block-content-based-on-java-reader-and-silverlight/
    recognize a dialup? i.imgur.com/Q3lKIr1.jpg
    http://www.johndcook.com/blog/2013/02/21/can-regular-expressions-parse-html-or-not/
    ``evil'' code: http://erratasec.blogspot.ca/2013/03/the-debate-over-evil-code.html



%\section{TBD}
%\subsection{Timeline}
%Contains a timeline of the work
%history; overview; phase 1, 2, 3
%\section{Themes}
%What are the major themes in play here?

%\begin{enumerate}
%  \item relations to other areas
%  \item systems work
%  \item theory work
%  \item tbd
%\end{enumerate}

%\section{Approach/Domain 1}
%papers in this domain do X.
%\section{Approach/Domain 2}

%\section{Summary}
%Major challenges.
%Where are we headed?
%FIN.

\bibliographystyle{acmtrans}
\bibliography{langsec}
\begin{received}
Received MM YYYY;
Revised MM YYYYY;
Accepted MM YYYYYY;
\end{received}

\end{document}


